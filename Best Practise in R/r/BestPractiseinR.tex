% Options for packages loaded elsewhere
\PassOptionsToPackage{unicode}{hyperref}
\PassOptionsToPackage{hyphens}{url}
%
\documentclass[
  ignorenonframetext,
]{beamer}
\usepackage{pgfpages}
\setbeamertemplate{caption}[numbered]
\setbeamertemplate{caption label separator}{: }
\setbeamercolor{caption name}{fg=normal text.fg}
\beamertemplatenavigationsymbolsempty
% Prevent slide breaks in the middle of a paragraph
\widowpenalties 1 10000
\raggedbottom
\setbeamertemplate{part page}{
  \centering
  \begin{beamercolorbox}[sep=16pt,center]{part title}
    \usebeamerfont{part title}\insertpart\par
  \end{beamercolorbox}
}
\setbeamertemplate{section page}{
  \centering
  \begin{beamercolorbox}[sep=12pt,center]{part title}
    \usebeamerfont{section title}\insertsection\par
  \end{beamercolorbox}
}
\setbeamertemplate{subsection page}{
  \centering
  \begin{beamercolorbox}[sep=8pt,center]{part title}
    \usebeamerfont{subsection title}\insertsubsection\par
  \end{beamercolorbox}
}
\AtBeginPart{
  \frame{\partpage}
}
\AtBeginSection{
  \ifbibliography
  \else
    \frame{\sectionpage}
  \fi
}
\AtBeginSubsection{
  \frame{\subsectionpage}
}
\usepackage{amsmath,amssymb}
\usepackage{lmodern}
\usepackage{ifxetex,ifluatex}
\ifnum 0\ifxetex 1\fi\ifluatex 1\fi=0 % if pdftex
  \usepackage[T1]{fontenc}
  \usepackage[utf8]{inputenc}
  \usepackage{textcomp} % provide euro and other symbols
\else % if luatex or xetex
  \usepackage{unicode-math}
  \defaultfontfeatures{Scale=MatchLowercase}
  \defaultfontfeatures[\rmfamily]{Ligatures=TeX,Scale=1}
\fi
\usetheme[]{spacelab}
% Use upquote if available, for straight quotes in verbatim environments
\IfFileExists{upquote.sty}{\usepackage{upquote}}{}
\IfFileExists{microtype.sty}{% use microtype if available
  \usepackage[]{microtype}
  \UseMicrotypeSet[protrusion]{basicmath} % disable protrusion for tt fonts
}{}
\makeatletter
\@ifundefined{KOMAClassName}{% if non-KOMA class
  \IfFileExists{parskip.sty}{%
    \usepackage{parskip}
  }{% else
    \setlength{\parindent}{0pt}
    \setlength{\parskip}{6pt plus 2pt minus 1pt}}
}{% if KOMA class
  \KOMAoptions{parskip=half}}
\makeatother
\usepackage{xcolor}
\IfFileExists{xurl.sty}{\usepackage{xurl}}{} % add URL line breaks if available
\IfFileExists{bookmark.sty}{\usepackage{bookmark}}{\usepackage{hyperref}}
\hypersetup{
  pdftitle={Best Practise in R},
  pdfauthor={Gareth Burns},
  hidelinks,
  pdfcreator={LaTeX via pandoc}}
\urlstyle{same} % disable monospaced font for URLs
\newif\ifbibliography
\usepackage{color}
\usepackage{fancyvrb}
\newcommand{\VerbBar}{|}
\newcommand{\VERB}{\Verb[commandchars=\\\{\}]}
\DefineVerbatimEnvironment{Highlighting}{Verbatim}{commandchars=\\\{\}}
% Add ',fontsize=\small' for more characters per line
\usepackage{framed}
\definecolor{shadecolor}{RGB}{248,248,248}
\newenvironment{Shaded}{\begin{snugshade}}{\end{snugshade}}
\newcommand{\AlertTok}[1]{\textcolor[rgb]{0.94,0.16,0.16}{#1}}
\newcommand{\AnnotationTok}[1]{\textcolor[rgb]{0.56,0.35,0.01}{\textbf{\textit{#1}}}}
\newcommand{\AttributeTok}[1]{\textcolor[rgb]{0.77,0.63,0.00}{#1}}
\newcommand{\BaseNTok}[1]{\textcolor[rgb]{0.00,0.00,0.81}{#1}}
\newcommand{\BuiltInTok}[1]{#1}
\newcommand{\CharTok}[1]{\textcolor[rgb]{0.31,0.60,0.02}{#1}}
\newcommand{\CommentTok}[1]{\textcolor[rgb]{0.56,0.35,0.01}{\textit{#1}}}
\newcommand{\CommentVarTok}[1]{\textcolor[rgb]{0.56,0.35,0.01}{\textbf{\textit{#1}}}}
\newcommand{\ConstantTok}[1]{\textcolor[rgb]{0.00,0.00,0.00}{#1}}
\newcommand{\ControlFlowTok}[1]{\textcolor[rgb]{0.13,0.29,0.53}{\textbf{#1}}}
\newcommand{\DataTypeTok}[1]{\textcolor[rgb]{0.13,0.29,0.53}{#1}}
\newcommand{\DecValTok}[1]{\textcolor[rgb]{0.00,0.00,0.81}{#1}}
\newcommand{\DocumentationTok}[1]{\textcolor[rgb]{0.56,0.35,0.01}{\textbf{\textit{#1}}}}
\newcommand{\ErrorTok}[1]{\textcolor[rgb]{0.64,0.00,0.00}{\textbf{#1}}}
\newcommand{\ExtensionTok}[1]{#1}
\newcommand{\FloatTok}[1]{\textcolor[rgb]{0.00,0.00,0.81}{#1}}
\newcommand{\FunctionTok}[1]{\textcolor[rgb]{0.00,0.00,0.00}{#1}}
\newcommand{\ImportTok}[1]{#1}
\newcommand{\InformationTok}[1]{\textcolor[rgb]{0.56,0.35,0.01}{\textbf{\textit{#1}}}}
\newcommand{\KeywordTok}[1]{\textcolor[rgb]{0.13,0.29,0.53}{\textbf{#1}}}
\newcommand{\NormalTok}[1]{#1}
\newcommand{\OperatorTok}[1]{\textcolor[rgb]{0.81,0.36,0.00}{\textbf{#1}}}
\newcommand{\OtherTok}[1]{\textcolor[rgb]{0.56,0.35,0.01}{#1}}
\newcommand{\PreprocessorTok}[1]{\textcolor[rgb]{0.56,0.35,0.01}{\textit{#1}}}
\newcommand{\RegionMarkerTok}[1]{#1}
\newcommand{\SpecialCharTok}[1]{\textcolor[rgb]{0.00,0.00,0.00}{#1}}
\newcommand{\SpecialStringTok}[1]{\textcolor[rgb]{0.31,0.60,0.02}{#1}}
\newcommand{\StringTok}[1]{\textcolor[rgb]{0.31,0.60,0.02}{#1}}
\newcommand{\VariableTok}[1]{\textcolor[rgb]{0.00,0.00,0.00}{#1}}
\newcommand{\VerbatimStringTok}[1]{\textcolor[rgb]{0.31,0.60,0.02}{#1}}
\newcommand{\WarningTok}[1]{\textcolor[rgb]{0.56,0.35,0.01}{\textbf{\textit{#1}}}}
\usepackage{graphicx}
\makeatletter
\def\maxwidth{\ifdim\Gin@nat@width>\linewidth\linewidth\else\Gin@nat@width\fi}
\def\maxheight{\ifdim\Gin@nat@height>\textheight\textheight\else\Gin@nat@height\fi}
\makeatother
% Scale images if necessary, so that they will not overflow the page
% margins by default, and it is still possible to overwrite the defaults
% using explicit options in \includegraphics[width, height, ...]{}
\setkeys{Gin}{width=\maxwidth,height=\maxheight,keepaspectratio}
% Set default figure placement to htbp
\makeatletter
\def\fps@figure{htbp}
\makeatother
\setlength{\emergencystretch}{3em} % prevent overfull lines
\providecommand{\tightlist}{%
  \setlength{\itemsep}{0pt}\setlength{\parskip}{0pt}}
\setcounter{secnumdepth}{-\maxdimen} % remove section numbering
\ifluatex
  \usepackage{selnolig}  % disable illegal ligatures
\fi

\title{Best Practise in R}
\author{Gareth Burns}
\date{02/12/2021}

\begin{document}
\frame{\titlepage}

\begin{frame}[allowframebreaks]
  \tableofcontents[hideallsubsections]
\end{frame}
\begin{frame}{Best Practise in R}
\protect\hypertarget{best-practise-in-r}{}
R is an open source ecosystem with vast array of different packages from
many different authors, evolving of over many years and across multiple
different disciples. There are no agreed conventions, even within base
R.

This makes it difficult for a R user to know what convention to adopt.

\emph{Aim:} To \emph{stimulate} discussion on best practise,
\emph{signpost} to resources \& \emph{suggest} some approaches to take.
\end{frame}

\begin{frame}{Key messages}
\protect\hypertarget{key-messages}{}
The key take home messages from the \emph{Jumping Rivers} course were:

\begin{itemize}
\tightlist
\item
  Write code for humans to understand, not computers
\item
  Best Practise is subjective and specific to individual use-cases
\item
  Adopt conventions, document them and be \emph{consistent} within a
  Team
\item
  Adopting best practice may not save you time short term but will save
  time in long term
\end{itemize}
\end{frame}

\begin{frame}{Naming Conventions}
\protect\hypertarget{naming-conventions}{}
Code will run without a standard naming convention but it will make it
difficult for others to read and interpret the code. It can lead to
bugs, difficulty to communicate what the code does and inefficiencies.\\
In R we name: - Files - Functions - Arguments - Variables - Classes
\end{frame}

\begin{frame}{Naming Conventions Examples}
\protect\hypertarget{naming-conventions-examples}{}
\end{frame}

\begin{frame}{Style Guidelines}
\protect\hypertarget{style-guidelines}{}
\begin{itemize}
\tightlist
\item
  Put space after a \emph{comma} and not before
\item
  Put space before and after an \emph{equals} sign
\item
  Use assignment operator (\textless-) for created functions and
  variables
\item
  Maximum 80 characters per line
\item
  Indent curly brackets (RStudio: \emph{Crtl + Shift + A})
\end{itemize}
\end{frame}

\begin{frame}{Modularise Code}
\protect\hypertarget{modularise-code}{}
\end{frame}

\begin{frame}[fragile]{Lessons Learnt 1}
\protect\hypertarget{lessons-learnt-1}{}
The precedence of a function will depend on the order you load packages.
You can call a function from a specific package using the syntax
\textgreater{} package\_name::function\_name

\begin{verbatim}
## 
## Attaching package: 'dplyr'
\end{verbatim}

\begin{verbatim}
## The following objects are masked from 'package:stats':
## 
##     filter, lag
\end{verbatim}

\begin{verbatim}
## The following objects are masked from 'package:base':
## 
##     intersect, setdiff, setequal, union
\end{verbatim}

\begin{verbatim}
## Warning in library(): libraries '/usr/local/lib/R/site-library', '/usr/lib/R/
## site-library' contain no packages
\end{verbatim}
\end{frame}

\begin{frame}[fragile]{Lessons Learnt 2}
\protect\hypertarget{lessons-learnt-2}{}
Don't \emph{hard-code} variables, create a variable at start of script.

\begin{Shaded}
\begin{Highlighting}[]
\FunctionTok{library}\NormalTok{(palmerpenguins)}
\FunctionTok{library}\NormalTok{(ggplot2)}

\NormalTok{gentooData }\OtherTok{\textless{}{-}}\NormalTok{ penguins[penguins}\SpecialCharTok{$}\NormalTok{species }\SpecialCharTok{==} \StringTok{"Gentoo"}\NormalTok{, ]}

\FunctionTok{ggplot}\NormalTok{(}\AttributeTok{data =}\NormalTok{ gentooData,}
       \FunctionTok{aes}\NormalTok{(}\AttributeTok{x =}\NormalTok{ flipper\_length\_mm,}
           \AttributeTok{y =}\NormalTok{ body\_mass\_g)) }\SpecialCharTok{+}
  \FunctionTok{geom\_point}\NormalTok{(}\FunctionTok{aes}\NormalTok{(}\AttributeTok{color =} \StringTok{"Red"}\NormalTok{),}
    \AttributeTok{size =} \DecValTok{3}\NormalTok{ ) }\SpecialCharTok{+}
      \FunctionTok{labs}\NormalTok{(}
        \AttributeTok{title =} \StringTok{"Relationship between flipper length and body mass for Gentoo Penguins"}\NormalTok{,}
        \AttributeTok{caption =} \StringTok{"Data Source: Palmers Penguins"}\NormalTok{,}
        \AttributeTok{x =} \StringTok{"Flipper length (mm)"}\NormalTok{,}
        \AttributeTok{y =} \StringTok{"Body mass (g)"}\NormalTok{,}
\NormalTok{      ) }\SpecialCharTok{+}
      \FunctionTok{theme\_minimal}\NormalTok{(}
\NormalTok{      )}
\end{Highlighting}
\end{Shaded}
\end{frame}

\begin{frame}[fragile]{Lessons Learnt 2}
\protect\hypertarget{lessons-learnt-2-1}{}
\begin{verbatim}
## Warning: Removed 1 rows containing missing values (geom_point).
\end{verbatim}

\includegraphics{BestPractiseinR_files/figure-beamer/palmer_penguins_plot-1.pdf}
\end{frame}

\begin{frame}[fragile]{Lessons Learnt 2}
\protect\hypertarget{lessons-learnt-2-2}{}
Copy and pasting code is a common practice but but can lead to errors
that aren't always picked up on.

\begin{Shaded}
\begin{Highlighting}[]
\FunctionTok{library}\NormalTok{(palmerpenguins)}
\FunctionTok{library}\NormalTok{(ggplot2)}

\NormalTok{chinstrapData }\OtherTok{\textless{}{-}}\NormalTok{ penguins[penguins}\SpecialCharTok{$}\NormalTok{species }\SpecialCharTok{==} \StringTok{"Chinstrap"}\NormalTok{, ]}

\FunctionTok{ggplot}\NormalTok{(}\AttributeTok{data =}\NormalTok{ chinstrapData,}
       \FunctionTok{aes}\NormalTok{(}\AttributeTok{x =}\NormalTok{ flipper\_length\_mm,}
           \AttributeTok{y =}\NormalTok{ body\_mass\_g)) }\SpecialCharTok{+}
  \FunctionTok{geom\_point}\NormalTok{(}\FunctionTok{aes}\NormalTok{(}\AttributeTok{color =} \StringTok{"Red"}\NormalTok{),}
    \AttributeTok{size =} \DecValTok{3}\NormalTok{ ) }\SpecialCharTok{+}
      \FunctionTok{labs}\NormalTok{(}
        \AttributeTok{title =} \StringTok{"Relationship between flipper length and body mass for Gentoo Penguins"}\NormalTok{,}
        \AttributeTok{caption =} \StringTok{"Data Source: Palmers Penguins"}\NormalTok{,}
        \AttributeTok{x =} \StringTok{"Flipper length (mm)"}\NormalTok{,}
        \AttributeTok{y =} \StringTok{"Body mass (g)"}
\NormalTok{      ) }\SpecialCharTok{+}
      \FunctionTok{theme\_minimal}\NormalTok{()}
\end{Highlighting}
\end{Shaded}
\end{frame}

\begin{frame}{Lessons Learnt 2}
\protect\hypertarget{lessons-learnt-2-3}{}
\end{frame}

\begin{frame}[fragile]{Lessions Learnt 2}
\protect\hypertarget{lessions-learnt-2}{}
Create a variable at the start of the script mitigates this risk and
makes it easier to change the scope of the code.

\begin{verbatim}
## Warning: Removed 1 rows containing missing values (geom_point).
\end{verbatim}

\includegraphics{BestPractiseinR_files/figure-beamer/palmer_penguins_plot_suggested-1.pdf}
\end{frame}

\begin{frame}{Lessons Learnt 3}
\protect\hypertarget{lessons-learnt-3}{}
Create a script to carry out as much tasks as possible or document where
externalities occur. Ideally if you're obtaining external data, write a
script for sourcing the data and save it in the same project file
structure. If the data was obtained via USB, e-mail or one off transfer
- document this. As the next person coming along will have no way of
knowing where these data were obtained from!
\end{frame}

\begin{frame}[fragile]{Lessons Learnt 4}
\protect\hypertarget{lessons-learnt-4}{}
When creating an index for referencing (such as a loop or lapply), use
\textbf{seq\_len}. Other sequence index creating methods can create a
sequence backwards and make it difficult to debug loops.

\begin{Shaded}
\begin{Highlighting}[]
\NormalTok{negativeNumber }\OtherTok{\textless{}{-}} \SpecialCharTok{{-}}\DecValTok{4}
\NormalTok{zero }\OtherTok{\textless{}{-}} \DecValTok{0}

\DecValTok{1}\SpecialCharTok{:}\NormalTok{negativeNumber}
\end{Highlighting}
\end{Shaded}

\begin{verbatim}
## [1]  1  0 -1 -2 -3 -4
\end{verbatim}

\begin{Shaded}
\begin{Highlighting}[]
\DecValTok{1}\SpecialCharTok{:}\NormalTok{zero}
\end{Highlighting}
\end{Shaded}

\begin{verbatim}
## [1] 1 0
\end{verbatim}

\begin{Shaded}
\begin{Highlighting}[]
\FunctionTok{seq}\NormalTok{(negativeNumber)}
\end{Highlighting}
\end{Shaded}

\begin{verbatim}
## [1]  1  0 -1 -2 -3 -4
\end{verbatim}

\begin{Shaded}
\begin{Highlighting}[]
\FunctionTok{seq}\NormalTok{(zero)}
\end{Highlighting}
\end{Shaded}

\begin{verbatim}
## [1] 1 0
\end{verbatim}

\begin{Shaded}
\begin{Highlighting}[]
\CommentTok{\# These will return an error before your loop is run}
\FunctionTok{seq\_len}\NormalTok{(negativeNumber)}
\FunctionTok{seq\_len}\NormalTok{(zero)}
\end{Highlighting}
\end{Shaded}
\end{frame}

\begin{frame}[fragile]{Lessons Learnt 5}
\protect\hypertarget{lessons-learnt-5}{}
Use \textbf{TRUE} \& \textbf{FALSE} not \textbf{T} \& \textbf{F}.
\textbf{TRUE} \& \textbf{FALSE} are protected terms in R. \textbf{T} \&
\textbf{F} are not.

\begin{Shaded}
\begin{Highlighting}[]
\NormalTok{T }\OtherTok{\textless{}{-}} \ConstantTok{FALSE}

\CommentTok{\# NA\textquotesingle{}s propagate}
\FunctionTok{mean}\NormalTok{(penguins}\SpecialCharTok{$}\NormalTok{bill\_length\_mm)}
\end{Highlighting}
\end{Shaded}

\begin{verbatim}
## [1] NA
\end{verbatim}

\begin{Shaded}
\begin{Highlighting}[]
\CommentTok{\# Change the na.rm argument to TRUE by using shorthand T}
\FunctionTok{mean}\NormalTok{(penguins}\SpecialCharTok{$}\NormalTok{bill\_length\_mm, }\AttributeTok{na.rm =}\NormalTok{ T)}
\end{Highlighting}
\end{Shaded}

\begin{verbatim}
## [1] NA
\end{verbatim}

\begin{Shaded}
\begin{Highlighting}[]
\FunctionTok{isFALSE}\NormalTok{(T)}
\end{Highlighting}
\end{Shaded}

\begin{verbatim}
## [1] TRUE
\end{verbatim}

\begin{Shaded}
\begin{Highlighting}[]
\FunctionTok{mean}\NormalTok{(penguins}\SpecialCharTok{$}\NormalTok{bill\_length\_mm, }\AttributeTok{na.rm =} \ConstantTok{TRUE}\NormalTok{)}
\end{Highlighting}
\end{Shaded}

\begin{verbatim}
## [1] 43.92193
\end{verbatim}

\begin{Shaded}
\begin{Highlighting}[]
\CommentTok{\# TRUE is a protected term in R}
\ConstantTok{TRUE} \OtherTok{\textless{}{-}} \ConstantTok{FALSE}
\end{Highlighting}
\end{Shaded}
\end{frame}

\begin{frame}{Lessons Learnt 6}
\protect\hypertarget{lessons-learnt-6}{}
Running a script will automatically overwrite your output files! If you
copy and paste a script and start to modify it this can lead to
unintended consequences! A simple wrapper function can check if a file
exists and if so append a date stamp suffix to prevent overwriting.
\end{frame}

\begin{frame}{Lessons Learnt 7}
\protect\hypertarget{lessons-learnt-7}{}
Use \textbf{isTRUE} or \textbf{isFALSE} for conditionals (e.g.~if
statements). If you supply more than one logical argument to
\end{frame}

\begin{frame}{Resources}
\protect\hypertarget{resources}{}
\begin{itemize}
\tightlist
\item
  Jumping Rivers Course Notes
\item
  Tidyverse style guide \url{https://style.tidyverse.org/}
\item
  Existing Packages
\item
  Each other
\end{itemize}
\end{frame}

\begin{frame}[fragile]{Slide with R Output}
\protect\hypertarget{slide-with-r-output}{}
\begin{Shaded}
\begin{Highlighting}[]
\FunctionTok{summary}\NormalTok{(cars)}
\end{Highlighting}
\end{Shaded}

\begin{verbatim}
##      speed           dist       
##  Min.   : 4.0   Min.   :  2.00  
##  1st Qu.:12.0   1st Qu.: 26.00  
##  Median :15.0   Median : 36.00  
##  Mean   :15.4   Mean   : 42.98  
##  3rd Qu.:19.0   3rd Qu.: 56.00  
##  Max.   :25.0   Max.   :120.00
\end{verbatim}
\end{frame}

\end{document}
